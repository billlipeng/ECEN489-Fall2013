\documentclass{article}

\usepackage[paper=letterpaper, margin=1in]{geometry}

\title{Motor Team Monday Deliverables}
\date{Friday, 4 October 2013}

\begin{document}
\maketitle

\section{Schemata}
\subsection{Input Side JSON}
The input side arduino will communicate with the input side control program by
passing simple JSON objects over the serial line.  These JSON objects will have
a single field, \texttt{tempK}, containing a scalar temperature value in Kelvin.

\subsection{Database}
The database will store readings from the input side and feedback from the
control side.  The input side data will be stored in a table named
\texttt{temp\_readings}, with columns \texttt{arduino\_id} (integer),
\texttt{reading\_time\_utc} (UTC timestamp), and \texttt{temp\_kelvin} (real
scalar value in Kelvin).  The control side data will be stored in a table names
\texttt{motor\_readings} with columns \texttt{arduino\_id} (integer),
\texttt{reading\_time\_utc} (UTC timestamp), and \texttt{motor\_volts} (real
scalar value in volts).

\section{Group Breakdown}
\begin{itemize}
  \item \emph{Input Side (Arduino and Control Program):} Michael Bass
  \item \emph{Visualization:} Taahir Ahmed, Narayanan
  \item \emph{Database:} Taahir Ahmed, Ashton, Desmond
  \item \emph{Control:} Dipanjan, Jeff, Clayton
\end{itemize}

\section{Database Credentials}
We have a database named \texttt{motor\_team} on \texttt{fulla.ece.tamu.edu}, with
user \texttt{motor\_team\_user} and password \texttt{motor\_team\_password}.
You will need to be on the campus network or have a VPN connection to campus to
connect to the database.

\section{Deliverables for Monday, 7 October 2013}
\begin{itemize}
\item \emph{Input Side:} Send JSON data from Arduino and unwrap it in the
  control program.
\item \emph{Visualization:} Produce a graph updating from some source.
\item \emph{Database:} Set up database.
\item \emph{Control:} Filtering and conditioning logic.

\end{itemize}

\end{document}