\documentclass{article}

\usepackage[paper=letterpaper, margin=2.5cm]{geometry}
\usepackage{minted}

\begin{document}
\textbf{\Large ECEN 489: Assignment 4}

\noindent\rule{\textwidth}{0.4pt}\\
Student Name: Taahir Ahmed\\
Student UIN: 119004623\\
\noindent\rule{\textwidth}{0.4pt}\\
Grader Name:\\
Grader UIN:\\
\noindent\rule{\textwidth}{0.4pt}\\

\section*{True or False}
\begin{enumerate}
  \item \textbf{0.5 pt} -- A same symbol can denote both a unary operator and a
    binary operator.

    True.

  \item \textbf{0.5 pt} -- A compile-time warning is always given when an
    overflow exception occurs.

    False.
\end{enumerate}

\section*{Short Questions}
\begin{enumerate}
  \item \textbf{1 pt} -- What is an overloaded operator?

    An overloaded operator is a user-defined operator for existing or
    user-defined types.  It permits assigning type-specific meaning to the
    standard C++ operators.  Care should be taken to not violate the semantics
    of the operators, though there is no language enforcement.

  \item \textbf{1 pt} -- Define a compound expression.
    
    A compound expression is an expression involving multiple operators and
    operands.
    
  \item \textbf{1 pt} -- Determine the results of the following expressions and
    test your answers by compiling the expressions.

    \begin{enumerate}
      \item \mint{cpp}|-40 * 4 + 23 / 7| = -157
      \item \mint{cpp}|-40 + 4 * 23 / 7| =  -27
      \item \mint{cpp}|40 / 4 * 23 % 7|  =    6
      \item \mint{cpp}|-40 / 4 * 23 % 7| =   -6
    \end{enumerate}

  \item \textbf{1 pt} -- What is the meaning of short-circuit evaluation in the
    context of logical operators?

    The compiler emits code that does not necessarily evaluate all
    subexpressions of a logical expression.  The terms at a given
    lexical level of the expression are evaluated from left to right,
    as written.  As soon as the truth or falsity of the subexpression
    can be determined, evaluation stops.  This allows subexpressions
    that have side-effects, such as the increment and decrement
    operators, to be conditionally invoked, which can be convenient in
    some situations.

  \item \textbf{1 pt} -- Explain the structure of the conditional operator.
      
    The conditional operator evaluates its condition (the expression to the left
    of the question mark).  If the condition is true, the conditional operator
    takes the value of the expression between the question mark and the colon.
    If the condition is false, the conditional operator takes the value of the
    expression to the right of the colon.

  \item \textbf{1 pt} -- Consider the following variables of the given type:
    \begin{enumerate}
      \item long
      \item unsigned long
      \item signed long
      \item double
      \item double
      \item unsigned int
      \item int
      \item unsigned long
    \end{enumerate}
\end{enumerate}

\section*{Programming Challenge}
\inputminted[bgcolor=lightgray]{bash}{summary-stats-output}

\end{document}
